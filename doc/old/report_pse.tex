\documentclass[]{IEEEtran}

\title{Modellazione in SystemC di un sistema hardware/software per il controllo della temperatura di una stanza}
\author{Stefano Centomo - VR123456}

\usepackage{graphicx}
\usepackage[italian]{babel}

\begin{document}
\maketitle

\begin{abstract}
Questo documento serve come traccia per la relazione da consegnare come primo assignment del corso di Progettazione di Sistemi Embedded. Si ricorda che la relazione va consegnata insieme al codice sorgente delle soluzioni.

Il sommario, o abstract, conterr\`a una brevissima descrizione degli obiettivi del progetto, delle caratteristiche principali del flusso di progettazione e dei risultati principali ottenuti. 
\end{abstract}

\section{Introduzione}

Per quanto riguarda le regole per il report:
\begin{itemize}
\item Il report pu\`o essere scritto sia in Italiano che in Inglese. Tuttavia, consiglio l'utilizzo dell'inglese solo nel caso abbiate gi\`a una base accettabile. Raccomando inoltre di scrivere in maniera ``decorosa'' e chiara: si tratta di un esame, non sono ammesse ``parole'' come ``cmq'' o ``xk\`e''.
\item \`E possibile utilizzare figure, tabelle e qualsiasi altro elemento che aiuti alla comprensione del report.

\item Il limite di pagine \`e posto a 10 per il formato a singola colonna e 8 per il formato a doppia colonna.
\end{itemize}

Nell'introduzione viene descritto in maniera astratta quello che poi viene dettagliato nel seguito del report. Una buona scaletta per l'introduzione pu\`o essere la seguente:
\begin{itemize}
\item Descrizione ad alto livello delle principali caratteristiche del sistema che si vuole modellare.
\item Descrizione delle motivazioni principali per l'utilizzo delle tecnologie descritte nel corso. Qual'\`e il problema che si vuole risolvere?
\item Descrizione dei passi utilizzati per arrivare all'implementazione finale. Descrivere la motivazione di ciascun passo. La descrizione dei passi dovrebbe formare la descrizione del flusso di lavoro svolto per completare l'assignment.
\item Rapidissima descrizione dei risultati principali.
\end{itemize}

L'introduzione non dovrebbe andare oltre la met\`a della seconda colonna (nel caso a due colonne), o la prima pagina (nel caso a colonna singola): bisogna cercare di essere concisi (e chiari). Alla fine, l'introduzione \`e solo ``chiacchiere'': deve semplicemente rendere chiari quali sono gli obiettivi del lavoro (\emph{e nel caso del corso, deve far capire a me che avete gli obiettivi chiari in testa}). Consiglio: l'introduzione (e spesso l'abstract) \`e l'ultima parte che viene completata.

\section{Background}

Il background dovrebbe contenere una descrizione, abbastanza breve, dei concetti principali che vengono utilizzati nel lavoro. Ad esempio, pu\`o contenere una breve descrizione delle caratteristiche principali di SystemC, e delle sue estensioni. Una colonna dovrebbe bastare.

In questa sessione saranno citati anche lavori dello stato dell'arte. Ad esempio, si pu\`o usare~\cite{SystemC} per citare SystemC. \textbf{I riferimenti bibliografici vanno inseriti ogni volta in cui si va a citare qualcosa contenuto in un documento.}

Questa sessione non deve ripetere quanto detto a lezione, ma dare un overview dei concetti principali utilizzati durante lo svolgimento dell'elaborato.

\section{Metodologia applicata}

In questa sezione viene descritto tutto il procedimento, ed \`e dunque la sezione pi\`u importante del report. Va descritto passo passo quello che avete fatto, facendo capire ``esattamente'' cosa \`e stato fatto.

Questa sessione pu\`o essere divisa in sottosezioni. Le informazioni riportate nella lista seguente dovrebbero essere identificabili nel testo del report (\textbf{anche in ordine diverso}):
\begin{itemize}
\item Definizione dell-architettura del programma: che parti scrivete in AMS, e che parti scrivette a tempo continuo? Quali sono i livelli di astrazione coinvolti?
\item Descrizione del componente di HW digitale, sua rappresentazione a TLM, e descrizione del processo di sintesi: TLM UT $\rightarrow$ TLM LT $\rightarrow$ TLM AT4. Il processo di sintesi devono giustificare i diversi algoritmi utilizzati ai diversi livelli di astrazione.
\item Processo di ``sintesi'' verso RTL. Definizione dei sottocomponenti del componente HW, e della sua struttura. Definizione dell'interfaccia RTL a partire dal TLM, definizione della Macchina a Stati Finiti Estesa (EFSM) del componente e dei sottocomponenti. Realizzazione del componente HW utilizzando i processi SystemC a livello RTL. \`E inoltre possibile discutere la scelta dei tipi di dato.
\item Descrizione della realizzazione della parte rappresentante SW Embedded, e descritta in TLM.
\item Descrizione della realizzazione della parte a tempo continuo. Spiegazione delle scelte progettuali fatte per gli stili di modellazione utilizzati.
\item Descrizione dei meccanismi di comunicazione tra le diverse parti del sistema.
\end{itemize}

\textbf{In questa sezione deve essere riportato (brevemente) anche l'organizzazione dell'implementazione consegnata assieme al report.}

\section{Risultati}

Qua vanno ``messi i numeri''. Questa sezione dovrebbe contenere i risultati della simulazione. La simulazione mostra che il sistema funziona correttamente? Come \`e stato provato? Che tipo di testbench sono stati utilizzati? Come \`e stato scomposto il sistema per verificarne la correttezza?

Per quanto riguarda le performance:
\begin{itemize}
\item cosa si pu\`o dire in merito ai tempi di simulazione dei modelli a diversi livelli di astrazione? Quali dati supportano questa affermazione? Come sono stati ottenuti i dati?
\item Come cambia la velocit\`a di simulazione per la parte digitale e la parte a tempo continuo? Come influiscono le frequenze scelte? E i livelli di astrazione? \emph{etc.}
\end{itemize}

Questa sezione pu\`o contenere anche riflessioni personali sui risultati ottenuti. Importante: tutte le affermazioni devono essere supportate da numeri\footnote{Richard Feynman on Scientific Method (1964) -\\ https://www.youtube.com/watch?v=OL6-x0modwY}.

\section{Conclusioni}
Le conclusioni dovrebbero riassumere in poche righe  tutto ci\`o che \`e stato fatto. Un paio di righe descrivono i risultati osservati, in modo da introdurre poi la conclusione ``vera e propria''. Nel caso del corso, la ``lezione da portare a casa'' sar\`a quello che si \`e imparato svolgendo l'elaborato.


\bibliographystyle{IEEEtran}
\bibliography{biblio}

\appendix
Se non avete abbastanza spazio, potete inserire le figure delle EFSM in una  pagina extra, appendice. Un esempio di come potete fare solo le Figure~\ref{fig:grande}, \ref{fig:piccola1}, \ref{fig:piccola2}.


\begin{figure*}[bt]
\centering
\includegraphics[width=\textwidth]{figures/EFSM}
\caption{Figura in formato grande.}
\label{fig:grande}
\end{figure*}

\begin{figure}[bt]
	\centering
	\includegraphics[width=\columnwidth]{figures/EFSM}
	\caption{Figura in formato piccolo, 1.}
	\label{fig:piccola1}
\end{figure}

\begin{figure}[bt]
	\centering
	\includegraphics[width=\columnwidth]{figures/EFSM}
	\caption{Figura in formato piccolo, 2.}
	\label{fig:piccola2}
\end{figure}

\end{document}