\section{Sintesi ad alto livello con Vivado HLS}

Proviamo ad importare il file di specifiche del modulo all'interno di \textsf{Vivado HLS}. L'unico file fornito viene diviso in un sorgente ed in un testbench (quest'ultimo contiene la sola funzione \code{main}). 

Per testare la correttezza delle specifiche avviamo la simulazione del codice C. Il risultato è corretto e ne segue il log (stampa del logo omessa):
\begin{minted}[fontsize=\footnotesize]{text}
INFO: [SIM 2] *************** CSIM start ***************
INFO: [SIM 4] CSIM will launch GCC as the compiler.
   Compiling ../../xtea_tb.cpp in debug mode
   Compiling ../../xtea.cpp in debug mode
   Generating csim.exe
First invocation: 
   - the encryption of 12345678 and 9abcdeff 
   - with key 6a1d78c88c86d67f2a65bfbeb4bd6e46 
   is: 99bbb92b, 3ebd1644 
Second invocation: 
   - the decryption of 99bbb92b and 3ebd1644 
   - with key 6a1d78c88c86d67f2a65bfbeb4bd6e46 
   is: 12345678, 9abcdeff 
Done!!
INFO: [SIM 1] CSim done with 0 errors.
INFO: [SIM 3] *************** CSIM finish ***************
\end{minted}

Infine avviamo la sintesi. In output abbiamo un file \code{xtea.vhd}. Questo viene importato in \textsf{Vivado} e sintetizzato. La sintesi ha esito positivo. L'implementazione ha esito negativo sempre a causa del numero di IO. 

\begin{table}[htbp]
    \centering
    \begin{tabular}{l S[table-format=3.0] S[table-format=3.0] }
        \toprule
        {Site Type}        	    & {VHDL RTL} & {Specifiche C++} \\ \midrule
 Slice LUTs              	    & 486	     &    696	 \\
 \quad LUT as Logic          	& 486	     &    696	 \\
 \quad LUT as Memory         	& 0	         &    0      \\
 Slice Registers         	    & 441	     &    606    \\
 \quad Register as Flip Flop 	& 15	     &    606    \\
 \quad Register as Latch     	& 426	     &    0      \\
 F7 Muxes                	    & 0	         &    0	     \\
 F8 Muxes                	    & 0	         &    0	     \\ \bottomrule
    \end{tabular}
    \caption{Confronto}
    \label{tab:confronto}
\end{table}

\section{Conclusioni}

La Tabella~\ref{tab:confronto} confronta l'utilizzo di componenti delle varie versioni del modulo.

Vediamo come la versione VHDL occupa circa il 15\% delle risorse (porte logiche, memoria) rispetto alla versione sintetizzata a partire dalle specifiche. La perdita di spazio è in realtà irrisoria rispetto al tempo perso per implementare il VHDL, quindi è abbastanza ovvio che questa prima versione è da preferirsi. 

Infine notiamo come la versione VHDL utilizzi un certo numero di latch invece che flip-flops, sintomo esiste un processo in cui uno o più segnali vengono scritti in alcuni branch mentre in altri no. Questo difetto è prodotto da una svista del programmatore. Nella versione sintetizzata questi problemi non compaiono.